\section{Appendix}

% Fand ich noch praktisch, hab ich von einer anderen ZF kopiert und ergänzt von unserer vorherigen ZF von Anal 1/2
\subsection{Trigonometry}
\begin{center}
    $\sinh(x)$ is odd and $\cosh(x)$ is even!
\end{center}

\begin{center}
    \renewcommand{\arraystretch}{1.5}
    \begin{tabular}{l l} \toprule
        \multicolumn{2}{l}{$\cos(x\pm y)=\cos(x)\cos(y)\mp\sin(x)\sin(y)$}                                                  \\
        \multicolumn{2}{l}{$\sin(x\pm y)=\sin(x)\cos(y)\pm\cos(x)\sin(y)$}                                                  \\
        \midrule
        $\cos \left(x+\frac{1}{2} \pi\right)=-\sin (x)$  & \hspace*{-10pt} $\sin \left(x+\frac{1}{2} \pi\right)=\cos (x)$   \\
        \midrule
        \multicolumn{2}{l}{$\sin(x)\sin(y)=\frac{1}{2}(\cos(x-y)-\cos(x+y))$}                                               \\
        \multicolumn{2}{l}{$\cos(x)\cos(y)=\frac{1}{2}(\cos(x-y)+\cos(x+y))$}                                               \\
        \multicolumn{2}{l}{$\sin(x)\cos(y)=\frac{1}{2}(\sin(x-y)+\sin(x+y))$}                                               \\
            \midrule
        $\sin^2(x)=\frac{1}{2}(1-\cos(2x))$              & \hspace*{-10pt} $\cos^2(x)=\frac{1}{2}(1+\cos(2x))$              \\
        $\sin^{3}(x)=\frac{1}{4}(3 \sin (x)-\sin (3 x))$ & \hspace*{-10pt} $\cos^{3}(x)=\frac{1}{4}(3 \cos (x)+\cos (3 x))$ \\
        \bottomrule
    \end{tabular}
\end{center}

\begin{center} 
    \begin{minipage}{0.47 \linewidth}
        \begin{center}
            \begin{tabular}{r l}
                $\sin(z)$ & \hspace*{-10pt}$= \dfrac{e^{iz} - e^{-iz}}{2i}$ \\[4mm]
                $\cos(z)$ & \hspace*{-10pt}$= \dfrac{e^{iz} + e^{-iz}}{2}$  \\
            \end{tabular}
        \end{center}
    \end{minipage}
    \begin{minipage}{0.47 \linewidth}
        \begin{center}
            \begin{tabular}{r l}
                $\sinh(z)$ & \hspace*{-10pt}$= \dfrac{e^{z} - e^{-z}}{2}$ \\[4mm]
                $\cosh(z)$ & \hspace*{-10pt}$= \dfrac{e^{z} + e^{-z}}{2}$ \\
            \end{tabular}
        \end{center}
    \end{minipage}
\end{center}

\begin{center}
    \renewcommand{\arraystretch}{1.25}
    \begin{tabular}{r c c c c c} \toprule
        deg/rad & 0$^\circ$/0 & 30$^\circ$/$\frac{\pi}{6}$ & $45^\circ$/$\frac{\pi}{4}$ & 60$^\circ$/$\frac{\pi}{3}$ & 90$^\circ$/$\frac{\pi}{2}$ \\ \midrule
        sin     & $0$         & $\frac{\sqrt{1}}{2}$       & $\frac{\sqrt{2}}{2}$       & $\frac{\sqrt{3}}{2}$       & $1$                        \\
        cos     & $1$         & $\frac{\sqrt{3}}{2}$       & $\frac{\sqrt{2}}{2}$       & $\frac{\sqrt{1}}{2}$       & $0$                        \\
        tan     & $0$         & $\frac{\sqrt{3}}{3}$       & $1$                        & $\sqrt{3}$                 & -                          \\ \bottomrule
    \end{tabular}
    \begin{tabular}{r c c c c} \toprule
        deg/rad & 120$^\circ$/$\frac{2\pi}{3}$ & 135$^\circ$/$\frac{3\pi}{4}$ & $150^\circ$/$\frac{5\pi}{6}$ & 180$^\circ$/$\pi$ \\ \midrule
        sin     & $\frac{\sqrt{3}}{2}$         & $\frac{\sqrt{2}}{2}$         & $\frac{1}{2}$                & $0$               \\
        cos     & $- \frac{1}{2}$              & $- \frac{\sqrt{2}}{2}$       & $- \frac{\sqrt{3}}{2}$       & $-1$              \\
        tan     & $-\sqrt{3}$                  & $- 1$         & $-\frac{\sqrt{3}}{3}$        & $0$               \\ \bottomrule
    \end{tabular}
\end{center}


\subsection{Fourier Series}
%copied from Koma cheatsheet

\textbf{Orthogonalitätsrelationen:}
$$ \frac{1}{2L} \int_{-L}^L e^{i \frac{n \pi}{L} t}e^{-i \frac{m \pi}{L} t} dt = \begin{cases}1&n=m\\ 0&n\ne m\end{cases}$$
$$ \int_{-L}^{L}\cos\left(\frac{n\pi t}{L}\right)\cos\left(\frac{m\pi t}{L}\right)dt=\begin{cases}0&n\ne m\\ L&n=m\ne0\\ 2L&n=m=0;\end{cases} $$
$$ \int_{-L}^{L}\sin\left(\frac{n\pi t}{L}\right)\sin\left(\frac{m\pi t}{L}\right)dt=\begin{cases}0&n\ne m\\ L&n=m\ne0;\end{cases} $$
$$ \int_{-L}^{L}\cos\left(\frac{n\pi t}{L}\right)\sin\left(\frac{m\pi t}{L}\right)dt=0, \text{ für alle n, m.} $$

% https://tutorial.math.lamar.edu/Classes/DE/PeriodicOrthogonal.aspx
\textbf{Orthogonalitätsrelationen auf $[0,L]$:}
$$ \int_0^L \cos\frac{n\pi t}{L}\cos\frac{m\pi t}{L}\ dt = \begin{cases}0 & n\ne m\\ \frac{L}{2} & n=m\ne0\\ L & n=m=0 \end{cases} $$
$$ \int_0^L \sin\frac{n\pi t}{L}\sin\frac{m\pi t}{L}\ dt = \begin{cases}0 & n\ne m\\ \frac{L}{2} & n=m \end{cases} $$
$$ \int_0^L \cos\frac{n\pi t}{L}\sin\frac{m\pi t}{L}\ dt = 0 \quad \text{für alle n, m} $$


\subsection{Derivatives and Antiderivatives}

\begin{center}
\renewcommand{\arraystretch}{1.3}
\begin{tabular}{|>{$}c<{$}|>{$}c<{$}|>{$}c<{$}|}
\hline
f'(x) & f(x) & \int f(x)\,dx \\
\hline
0 & c & cx \\
nx^{n-1} & x^n & \frac{x^{n+1}}{n+1} \\
-\frac{1}{x^2} & \frac{1}{x} & \ln|x| \\
\frac{n}{x^{n+1}} & \frac{1}{x^n} & \frac{-1}{(n-1)x^{n-1}} \\
\frac{1}{2\sqrt{x}} & \sqrt{x} & \frac{2}{3}x^{3/2} \\
ae^{ax} & e^{ax} & \frac{1}{a}e^{ax} \\
\frac{1}{x} & \ln|x| & x(\ln x - 1) \\
a^x \ln a & a^x & \frac{1}{\ln a} a^x \\
\cos(x) & \sin(x) & -\cos(x) \\
-\sin(x) & \cos(x) & \sin(x) \\
\frac{1}{\cos^2(x)} & \tan(x) & -\ln|\cos(x)| \\
\cosh(x) & \sinh(x) & \cosh(x) \\
\sinh(x) & \cosh(x) & \sinh(x) \\
\frac{1}{\cosh^2(x)} & \tanh(x) & \ln(\cosh(x)) \\
\frac{1}{1+x^2} & \arctan(x) & x\arctan(x) - \frac{\ln(x^2+1)}{2}\\
\frac{1}{\sqrt{1 - x^2}} & \arcsin(x) & x \arcsin(x) + \sqrt{1-x^2}\\
-\frac{1}{\sqrt{1 - x^2}} & \arccos(x) & x \arccos(x) - \sqrt{1-x^2}\\
\frac{1}{\sqrt{1 + x^2}} & \text{arcsinh}(x) & x \, \text{arcsinh}(x) - \sqrt{x^2+1} \\
\frac{1}{\sqrt{x^2 - 1}} \quad x \geq 1 & \text{arccosh}(x) & x \, \text{arccosh}(x) - \sqrt{x^2 - 1} \\
\frac{1}{1 - x^2} \quad |x| < 1 & \text{artanh}(x) & x \, \text{artanh}(x) + \frac{1}{2} \ln(1 - x^2) \\
\frac{1}{m((x+k)^2 + m^2)} & \frac{1}{(x+k)^2 + m^2} & \frac{1}{m} \arctan\left(\frac{x + k}{m}\right) \\
\hline
\end{tabular}
\end{center}
