\section{Preliminaries}
\textbf{ODE}: Ordinary differential equation, involving functions of one independent variable and one or more of their derivatives.

\textbf{PDE}: Partial differential equation, equation involving an unknown function of more than one variable and certain of its partial derivatives.

\textbf{Schwartz}: if $u$ has continuous second partial derivatives, then the order can be changed:
$$u_{xy} = u_{yx}$$

\textbf{Gradient}: of $u = u(x, y, z)$ is defined as
$$\nabla u := (u_x, u_y, u_z)$$
and the \textbf{Laplacian} of $u$ is
$$\Delta u := u_{xx} + u_{yy} + u_{zz}$$

\textbf{Problem}: A PDE supplemented with initial or boundary conditions. A problem is \textbf{well-posed} if it satisfies the following criteria:
\begin{enumerate}
  \item The problem has a solution (existence).
  \item The solution is unique (uniqueness).
  \item A small change in the equation and/or in the side conditions gives rise to a small change in the solution (stability).
\end{enumerate}
If one or more of these conditions do not hold, then the problem is said to be \textbf{ill-posed}.

\subsection{Classification properties of PDEs}
\textbf{order} of a PDE is the order of the highest order partial derivative of the unknown appearing within it.

$\implies u_{xyz} = xy^2$ has order 2 

\textbf{a linear} PDE is of the form
$$
a^{(0)}u + \sum_{i_1=1}^n a^{(1)}_{i_1} u_{x_{i_1}}
+ \sum_{i_1,i_2=1}^n a^{(2)}_{i_1,i_2} u_{x_{i_1}x_{i_2}} + \dots
=:\mathcal{L}[u] = f(\mathbf{x})
$$

$\implies uu_x=2$ is not linear, but $u_{tt} = u_{xxxxx}$ is

\textbf{homogenous} PDEs have $f(x)=0$

A PDE is \textbf{quasilinear} if it is linear in its highest order derivative term.

$\implies uu_x + u^2u_y + u = e^4$ is quasilinear (of first order)

\subsection{Strong and weak solutions}
\begin{itemize}
    \item \textbf{Strong solution (or classical solution):} All derivatives that appear in the PDE exist and are continuous on the whole domain of the PDE.
    \item \textbf{Weak solution:} There are points in the domain where the derivatives do not exist (or are not continuous). A weak solution cannot directly be plugged into the equation.
\end{itemize}

\section{Method of Characteristics (MoC)}

An approach to solving 1st order quasilinear PDEs.

\[
\begin{cases}
a \cdot u_x + b \cdot u_y = c(x,y,u) \\
u(x_0,y_0) = \tilde{u}_0
\end{cases}
\quad
\text{1) Initial curve: } \Gamma(s) =
\begin{pmatrix}
x_0(s) \\
y_0(s) \\
\tilde{u}_0(s)
\end{pmatrix}
\]

2) The \textit{characteristic equations} of the PDE are given by:

\[
\begin{array}{c|c}
\textbf{ODEs} & \textbf{Initial conditions} \\ \hline
\dfrac{d}{dt} x(t,s) = a(x,y,u) & x(0,s) = x_0(s) \\[1em]
\dfrac{d}{dt} y(t,s) = b(x,y,u) & y(0,s) = y_0(s) \\[1em]
\dfrac{d}{dt} \tilde{u}(t,s) = c(x,y,u) & \tilde{u}(0,s) = \tilde{u}_0(s)
\end{array}
\]

The resulting \textit{characteristic curves} span the solution surface.

3) Find \(t(x,y)\) and \(s(x,y)\), then:  
\[
u(x,y) = \tilde{u}(t(x,y), s(x,y))
\]

\subsection{Existence and uniqueness}

\section{Conservation laws and shock waves}

\section{One dimensional wave equation}

\section{Seperation of variables}

\section{Elliptic equations}

\section{Maximum principles}

\section{Laplace's equation in rectangular and circular domains}

