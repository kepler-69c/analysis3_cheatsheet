% content/summary.tex

\section{Preliminaries}
\textbf{ODE}: Ordinary differential equation, involving functions of one independent variable and one or more of their derivatives.

\textbf{PDE}: Partial differential equation, equation involving an unknown function of more than one variable and certain of its partial derivatives.

\textbf{Schwartz}: if $u$ has continuous second partial derivatives, then the order can be changed: $u_{xy} = u_{yx}$

\begin{align*}
  \text{For}~u = u(x, y, z): \quad  & \nabla u := (u_x, u_y, u_z) & \text{\textbf{Gradient}} \\
  & \Delta u := u_{xx} + u_{yy} + u_{zz} & \text{\textbf{Laplacian}}
\end{align*}

\textbf{Problem}: A PDE supplemented with initial or boundary conditions. A problem is \textbf{well-posed} if it satisfies the following criteria:
\begin{enumerate}
  \item The problem has a solution (existence).
  \item The solution is unique (uniqueness).
  \item A small change in the equation and/or in the side conditions gives rise to a small change in the solution (stability).
\end{enumerate}
If one or more of these conditions do not hold, then the problem is said to be \textbf{ill-posed}.

\subsection{Classification properties of PDEs}
\textbf{order} of a PDE is the order of the highest order partial derivative of the unknown appearing within it. $\implies u_{xz} = xy^2$ has order 2

\textbf{a linear} PDE is of the form
$$
a^{(0)}u + \sum_{i_1=1}^n a^{(1)}_{i_1} u_{x_{i_1}}
+ \sum_{i_1,i_2=1}^n a^{(2)}_{i_1,i_2} u_{x_{i_1}x_{i_2}} + \dots
=:\mathcal{L}[u] = f(\mathbf{x})
$$

$\implies uu_x=2$ is not linear, but $u_{tt} = u_{xxxxx}$ is

\textbf{homogenous} PDEs have $f(x)=0$

A PDE is \textbf{quasilinear} if it is linear in its highest order derivative term.

$\implies uu_x + u^2u_y + u = e^4$ is quasilinear (of first order)

\subsection{Strong and weak solutions}
\begin{itemize}
  \item \textbf{Strong solution (or classical solution):} All derivatives that appear in the PDE exist and are continuous on the whole domain of the PDE.
  \item \textbf{Weak solution:} There are points in the domain where the derivatives do not exist (or are not continuous). A weak solution cannot directly be plugged into the equation.
\end{itemize}

$\to$ Weak solutions are NOT unique.

\section{Method of Characteristics (MoC)}
Reduces first order / quasilinear PDEs by reducing them to a system of ODEs using geometrical interpretation.

The first two equations ($x_0(s),y_0(s)$) of the initial curve lie on the xy-plane. We use $s,t$ as a sort of "coordinate transformation"
\begin{recipe}[solve PDE using method of characteristics]
  Our PDE has the form $a(x,y,u)u_x + b(x,y,u)u_y = c(x,y,u)$
  \begin{enumerate}
    \item define $a$, $b$, $c$
    \item define \textbf{initial curve} based on the side condition(s)
      $$\varGamma(s) = (x_0(s), y_0(s), \tilde{u}_0(s)) = (\dots)$$
    \item solve the ODEs, using the initial values from (2):
      $$\frac{dx}{dt}=a \quad \frac{dy}{dt}=b \quad \frac{d\tilde{u}}{dt}=c$$
    \item The ODEs form a system of equations. Rearrange them according to t and s, then substitute the result into u(t) to obtain u(x, y).
  \end{enumerate}
\end{recipe}

\textbf{Example} (Method of characteristics)
$$xu_x+u_y = 1 \quad u(x,0)=f(x)$$

$\to$ define $a,b,c,\varGamma$

$$
\begin{pmatrix}
  a\\b\\c
\end{pmatrix} =
\begin{pmatrix}
  x\\1\\1
\end{pmatrix} \quad \varGamma(s) =
\begin{pmatrix}
  x_{0}(s)\\y_{0}(s) \\\tilde{u}_{0}(s)
\end{pmatrix} =
\begin{pmatrix}
  s\\0\\f(s)
\end{pmatrix}
$$

$\to$ solve ODEs

\begin{align*}
  \frac{dx}{dt} = x & &  \implies & x(t) = C_{1}e^t && \implies x(t)=se^t\\
  & &  & x(0) = C_{1} \stackrel{!}{=} s \\
  \frac{dy}{dt} = 1 &  & \implies  & y(t) = t+C_{2} && \implies y(t) = t\\
  &  &  & y(0) = C_{2} \stackrel{!}{=} 0 \\
  \frac{d\tilde{u}}{dt} = 1 &  & \implies & \tilde{u}(t) = t+C_{3} &  & \implies \tilde{u}(t) = t + f(s)\\
  &  &  & \tilde{u}(0) = C_{3} \stackrel{!}{=} f(s)
\end{align*}

$\to$ rearrange to x,y

$$
y=t;\quad\quad x=se^t = se^y \iff s=xe^{-y};\quad\quad \tilde{u}(t) = t+f(s) = y + f(xe^{-y})
$$

Thus, the result is $u(x,y) = y + f(xe^{-y})$.

\subsection{Existence and uniqueness}
Consider the Cauchy problem ($\Gamma\in C^1$ parametrizes initial curve in $xy$-plane):

\[
  \begin{cases}
    a(x, y, u)u_x + b(x, y, u)u_y = c(x, y, u) \\
    \Gamma(s) = (x_0(s),y_0(s),u_0(s))
  \end{cases}
\]
Assume that $\exists s_0 \in \mathbb{R}$ such that \textbf{transversality cond.} holds at $(0, s_0)$
\[
  \det
  \begin{pmatrix}
    \partial_{t} x(0,s_0) & \partial_{t} y(0,s_0) \\
    \partial_{s} x(0,s_0) & \partial_{s} y(0,s_0)
  \end{pmatrix} =
  \det
  \begin{pmatrix}
    a & b \\
    \partial_s x_0 & \partial_s y_0
  \end{pmatrix}
  \neq
  0
\]

Then there exists a unique solution u of the Cauchy problem in a neighbourhood
of $(x_0(s_0),y_0(s_0))$

\section{Conservation Laws}
First-order PDE describing the evolution of a density $u(x,t)$ under a flux $f(u)$:

$f:\mathbb{R}\to\mathbb{R}$ flux function, $c(u)=f'(u)$

$$u_y + f(u)_x = 0 \quad \text{or} \quad u_y + c(u)u_x = 0$$

where $c(u)$ represents the \textbf{characteristic speed}.

\subsection{Shock Waves and Weak Solutions}
\textbf{shock wave}: forms when characteristics intersect. Speed given by \textbf{Rankine-Hugoniot}:
$$\sigma'(y) = \frac{f(u^+) - f(u^-)}{u^+ - u^-}$$

% characteristic slope vs shock speed
\textbf{Characteristics}: The slope of the characteristic curves is given by $\frac{dx}{dy} = c(u)$. A shock exists where characteristics with different speeds $\alpha = c(u^-)$ and $\beta = c(u^+)$ collide.

\textbf{critical time} (note that $c(u)=f'(u)$):
$$
y_c = \inf\left\{ \frac{-1}{c'(u_0(s))u_0'(s)}: \quad s\in \mathbb{R},\ c'(u_{0}(s))u_{0}'(s)<0 \right\}
$$

A function is a weak solution $\iff$ \textbf{integral formulation} of conservation laws holds $\forall a<b,\ y_1<y_2$:
$$
\int_{a}^{b} u(x, y_{2}) \,dx - \int_{a}^{b} u(x, y_{1}) \,dx = - \int_{y_{1}}^{y_{2}} \left[ f(u(b, y)) - f(u(a, y)) \right] \,dy
$$

\textbf{entropy condition}: A weak solution satifies the entropy condition of characteristics only enter shocks but do not emanate from them. This means:
$$c(u^+) < \sigma' < c(u^-)$$

(this condition ensures uniqueness of weak solution)

\subsection{Burgers' equation.}
Models the flow of a mass with concentration $u(x,y)$ and initial
concentration $h(x)$ at $y=0$. The flow speed depends on the concentration.

\[
  \begin{cases}
    u_y + u\,u_x = 0,\\[2mm]
    u(x,0) = h(x),
  \end{cases}
  \qquad
  c(u) = u, \qquad
  f(u) = \tfrac12 u^2 .
\]

\subsection{Notion of weak solutions}
Let $f$ be the flux. The integral formulation of conservation laws
(also valid when $u$ is not smooth on the whole domain) is:
\[
  \int_a^b \bigl(u(x,y_2) - u(x,y_1)\bigr)\,dx
  = - \int_{y_1}^{y_2} \bigl[f(u(b,y)) - f(u(a,y))\bigr]\,dy .
\]

Let $D$ be the domain of $u(x,y)$, and suppose $D$ is divided into
subdomains $D_i$ such that $u$ is smooth on each $D_i$.

\begin{itemize}
  \item $u(x,y)$ is a \textbf{weak solution} on
    $D = \bigcup_{i=1}^n D_i$ if it is smooth on each subdomain $D_i$
    and satisfies the integral conservation law on $D$.
  \item The boundaries between the subdomains $D_i$ are curves called
    \textbf{shocks}.
\end{itemize}

\section{Classification of Second Order PDEs}
% the type of a second-order pde determines the qualitative behavior of its solutions
General linear second-order PDE in two variables ($a-g$ functions of $(x, y)$ but not $u$):
$$a u_{xx} + 2b u_{xy} + c u_{yy} + d u_x + e u_y + f u = g$$

The classification is determined by the sign of the \textbf{discriminant} $\delta = b^2 - ac$:
\begin{itemize}
  \item If $\delta > 0$: PDE is \textbf{hyperbolic} ($\sim$ Wave equation)
  \item If $\delta = 0$: PDE is \textbf{parabolic} ($\sim$ Heat equation)
  \item If $\delta < 0$: PDE is \textbf{elliptic} ($\sim$ Laplace equation)
\end{itemize}

% heat equation formula
\section{Heat equation}
\textbf{Standard form} ($k > 0$ thermal diffusivity):
$$u_t - k u_{xx} = f(x, t)$$

\textbf{Maximum Principle for the Heat Equation}:
For a solution $u(x, t)$ on the domain $[0, L] \times [0, T]$, the maximum (and minimum) values are always attained on the \textbf{parabolic boundary} $\Gamma_T$:
$$\Gamma_T = \{ (x, 0) : x \in [0, L] \} \cup \{ (0, t) : t \in [0, T] \} \cup \{ (L, t) : t \in [0, T] \}$$
This implies that if $f(x, t) \le 0$, the temperature cannot exceed its initial or boundary values.

\section{One dimensional wave equation}
Hyperbolic 2nd order differential equation of the form:
$$
\begin{cases}
  u_{tt} - c^2u_{xx} &= F(x,t) \\
  u(x,0) &= f(x) \\
  u_{t}(x,0) &= g(x)
\end{cases}
$$

\warning{$f\land g \land F$ odd/even/T-periodic in $x \implies$ $u(x,t)$ odd/even/T-periodic in $x$, too (see 6.3/7.2)}

% d'Alembert requirements: f in C^2, g in C^1, F in C^1
\textbf{Requirements}: For a classical solution, $f \in C^2(\mathbb{R})$, $g \in C^1(\mathbb{R})$, and $F \in C^1(\mathbb{R} \times [0, \infty))$.

\subsection{homogenous wave equation (F(x,t)=0)}
Solution consists of a backwards-moving wave $F$ and forward-moving wave $G$:
$$
u(x,t) = F(x+ct) + G(x-ct)
$$
The solution can be calculated using \textbf{d'Alembert's} (short) \textbf{formula}:
$$
u(x,t) = \frac{f(x+ct) + f(x-ct)}{2} + \frac{1}{2c} \int_{x-ct}^{x+ct}g(y)\ dy
$$

\subsection{non-homogenous wave equation}
\textbf{D'Alembert's formula} for non-homogenous wave equations:
$$
u(x,t) = \frac{f(x+ct) + f(x-ct)}{2} + \frac{1}{2c} \int_{x-ct}^{x+ct}g(y)\ dy + \frac{1}{2c}\int_0^t \int_{x-c(t-\tau)}^{x+c(t-\tau)} F(\xi, \tau) \ d\xi d\tau
$$

\begin{recipe}[trick: converting to homogenous wave equation]
  For a non-homogenous wave equation $u$, define $w:= u-v$, $w$ homogenous:
  $$
  \begin{cases}
    w_{tt}-c^2 w_{xx}  & =0 \\
    w(x,0) = u(x,0)-v(x,0)  & = f(x) - v(x,0) \\
    w_{t}(x,0) = u_{t}(x,0) - v_{t}(x,0)  & = g(x) - v_{t}(x,0)
  \end{cases}
  $$
  After finding $v$, we can now solve $w$ using d'Alembert's simple formula.

  % how to find v: usually a steady-state or particular solution
  \textbf{How to find $v$}: If boundary conditions are constant $u(0, t) = A$ and $u(L, t) = B$, choose $v(x) = A + \frac{B-A}{L}x$ (the linear steady-state solution). If there is a source $F(x)$, solve $-c^2 v''(x) = F(x)$.
\end{recipe}

\textbf{Example: 1D Wave Equation on a Ring (periodic extension)}

\[
  \begin{cases}
    u_{tt} - c^2 u_{xx} = 0, & x \in [0,1],\, t>0,\\[0.3em]
    u(x,0) = v_0(x), \quad u_t(x,0) = v_1(x),\\[0.3em]
    u(0,t) = u(1,t), \quad u_x(0,t) = u_x(1,t) \quad (\text{periodic BC})
  \end{cases}
\]

\[
  v_0(x) = x - x^2, \quad v_1(x) = 0, \quad c = \sqrt{\pi} - 1
\]

\vspace{0.3em}
\textit{D’Alembert solution:}
\[
  u(x,t) = \tfrac{1}{2}\big[v_0(x+ct) + v_0(x-ct)\big]
  + \tfrac{1}{2c}\int_{x-ct}^{x+ct} v_1(y)\,dy
\]

Since \(v_1(x)=0\):
\[
  u(x,t) = \tfrac{1}{2}\big[v_0(x+ct) + v_0(x-ct)\big]
\]

\textit{At } \(x=\tfrac{1}{2},\; t=2025\):
\[
  u\left(\tfrac{1}{2},2025\right)
  = \overbrace{u\left(\tfrac{1}{2},0\right)}^{\text{1-periodic}}
  = \tfrac{1}{2}\big[v_0(\tfrac{1}{2}+t) + v_0(\tfrac{1}{2}-t)\big]
  = v_0\!\left(\tfrac{1}{2}\right)
  = \tfrac{1}{2} - \left(\tfrac{1}{2}\right)^2
  = \tfrac{1}{4}.
\]

\warning{\textbf{Uniqueness} the 1D-wave equation has a unique solution.}

\section{Domain of dependence and region of influence}
\vspace{-10pt}
\begin{table}[H]
  \centering
  \begin{tabular}{c c}
    \includegraphics[width=0.45\linewidth]{img/region_of_influence.png}
    &
    \includegraphics[width=0.45\linewidth]{img/characteristictriangle.png}
    \\
    \scriptsize Region of influence
    &
    \scriptsize Characteristic triangle (domain of dependence)
  \end{tabular}
\end{table}

\textbf{Region of influence.}
The region of influence of an interval $[a,b]$ consists of all points $(x,t)$
with $t>0$ that can be affected by the initial data on $[a,b]$.
These points satisfy
\[
  x - ct \le b, \qquad x + ct \ge a.
\]
Outside this region, changes in $f$ or $g$ on $[a,b]$ do not influence the solution.

\medskip

\textbf{Domain of dependence.}
The value of the solution at a point $(x_0,t_0)$ depends only on the initial data in
\[
  [x_0 - ct_0,\; x_0 + ct_0].
\]
Information outside this interval cannot reach $(x_0,t_0)$.
The lines
\[
  x - ct = x_0 - ct_0, \qquad x + ct = x_0 + ct_0
\]
form the boundary of the \emph{characteristic triangle}
$\Delta_{(x_0,t_0)}$.

\medskip

\textbf{Smoothness.}
If the initial data is smooth on $[x_0 - ct_0,\; x_0 + ct_0]$, then the solution is smooth
inside the characteristic triangle $\Delta_{(x_0,t_0)}$.

\section{Separation of variables}
% compatibility condition
\textbf{Compatibility Conditions}: To ensure $u \in C^0(\overline{D})$, initial and boundary conditions must match at corners: $f(0) = u(0,0)$ and $f(L) = u(L,0)$. For $C^1$ or $C^2$ solutions, higher-order derivatives must be consistent.

\begin{recipe}[Separation of variables for heat/wave equation]
  \begin{enumerate}
    \item separation of variables to get two ODEs: $u(x,t) = X(x)T(t)$

      Substitute approach into PDE, transform to $\frac{X^{(n)}(x)}{X(x)} = \pm\frac{T^{(m)}(t)}{T(t)} = -\lambda$

      ODEs: $X^{(n)}(x) + \lambda X(x) = 0 \quad T^{(m)}(t) \mp \lambda T(t) = 0$

    \item solve ODEs
    \item use \textbf{boundary conditions} to determine eigenvalues $\lambda_n$ and eigenfunctions $X_n(x)$
    \item get general solution by superposition

      $$u(x,t) = \sum T_n(t) \cdot X_n(x)$$

    \item impose \textbf{initial conditions} to get coefficients of resulting (fourier series) solution
  \end{enumerate}
\end{recipe}

$\to$ Can be applied to "nearly all" linear 2nd-order PDEs.

\subsection{General solutions}

\vspace{-15pt}

\begin{table}[H]
  \scriptsize
  \setlength{\tabcolsep}{4pt}
  \renewcommand{\arraystretch}{1.15}

  \begin{tabular}{|m{0.02\linewidth}|m{0.31\linewidth}|m{0.57\linewidth}|}
    \hline
    \textbf{BC}
    & \textbf{Heat $u_t = k u_{xx}$}
    & \textbf{Wave $u_{tt} = c^2 u_{xx}$}
    \\ \hline

    \textbf{D}\\[-3pt]
    &
    $u=\displaystyle\sum_{n=1}^{\infty} b_n e^{-k(\frac{n\pi}{L})^2 t}\sin\!\left(\tfrac{n\pi x}{L}\right)$
    &
    $\displaystyle
    u=\sum_{n=1}^{\infty}
    \sin\!\left(\tfrac{n\pi x}{L}\right)
    \!\left[
      A_n\cos\!\left(\tfrac{n\pi c}{L}t\right)+
      B_n\sin\!\left(\tfrac{n\pi c}{L}t\right)
    \right]
    $
    \\ \hline

    \textbf{N}\\[-3pt]
    &
    $u=a_0 + \displaystyle\sum_{n=1}^{\infty}
    a_n e^{-k(\frac{n\pi}{L})^2 t}\cos\!\left(\tfrac{n\pi x}{L}\right)$
    &
    $\displaystyle
    u=A_0+
    \sum_{n=1}^{\infty}
    \cos\!\left(\tfrac{n\pi x}{L}\right)
    \!\left[
      A_n\cos\!\left(\tfrac{n\pi c}{L}t\right)+
      B_n\sin\!\left(\tfrac{n\pi c}{L}t\right)
    \right]
    $
    \\ \hline

    \textbf{DN}\\[-3pt]
    &
    $X_n=\sin\!\left(\tfrac{(n-\tfrac12)\pi x}{L}\right)\newline
    u=\displaystyle\sum_{n=1}^{\infty}
    c_n e^{-k((n-\tfrac12)\frac{\pi}{L})^2 t} X_n$
    &
    $\displaystyle
    u=\sum_{n=1}^{\infty} X_n
    \left[
      A_n\cos\!\left(\tfrac{(n-\tfrac12)\pi c}{L}t\right)
      +
      B_n\sin\!\left(\tfrac{(n-\tfrac12)\pi c}{L}t\right)
    \right]
    $
    \\ \hline

    \textbf{ND}\\[-3pt]
    &
    $X_n=\cos\!\left(\tfrac{(n-\tfrac12)\pi x}{L}\right)\newline
    u=\displaystyle\sum_{n=1}^{\infty}
    c_n e^{-k((n-\tfrac12)\frac{\pi}{L})^2 t} X_n$
    &
    $\displaystyle
    u=\sum_{n=1}^{\infty} X_n
    \left[
      A_n\cos\!\left(\tfrac{(n-\tfrac12)\pi c}{L}t\right)
      +
      B_n\sin\!\left(\tfrac{(n-\tfrac12)\pi c}{L}t\right)
    \right]
    $
    \\ \hline
  \end{tabular}
\end{table}
\vspace{-10pt}

{
  \scriptsize
  \textbf{Legend (Boundary Conditions):}

  \begin{itemize}[leftmargin=10pt, itemsep=0pt, topsep=0pt]

    \item \textbf{D} (Dirichlet): $u(0,t)=0,\;u(L,t)=0$
      \hspace*{12pt}$X_n(x)=\sin\!\left(\tfrac{n\pi x}{L}\right)$
      \hspace*{12pt}$\lambda_n=\left(\tfrac{n\pi}{L}\right)^2$

    \item \textbf{N} (Neumann): $u_x(0,t)=0,\;u_x(L,t)=0$
      \hspace*{12pt}$X_n(x)=\cos\!\left(\tfrac{n\pi x}{L}\right)$
      \hspace*{12pt}$\lambda_n=\left(\tfrac{n\pi}{L}\right)^2$

    \item \textbf{DN} (Mixed): $u(0,t)=0,\;u_x(L,t)=0$
      \hspace*{12pt}$X_n(x)=\sin\!\left(\tfrac{(n-\tfrac12)\pi x}{L}\right)$
      \hspace*{12pt}$\lambda_n=\left(\tfrac{(n-\tfrac12)\pi}{L}\right)^2$

    \item \textbf{ND} (Mixed): $u_x(0,t)=0,\;u(L,t)=0$
      \hspace*{12pt}$X_n(x)=\cos\!\left(\tfrac{(n-\tfrac12)\pi x}{L}\right)$
      \hspace*{12pt}$\lambda_n=\left(\frac{(n-\tfrac12)\pi}{L}\right)^2$

    \item \textbf{R} (Robin):
      \[
        \alpha_0 u(0,t) + \beta_0 u_x(0,t) = \gamma_0, \qquad
        \alpha_L u(L,t) + \beta_L u_x(L,t) = \gamma_L
      \]

  \end{itemize}

}

\begin{itemize}
  \item $\lambda_n = \left( \frac{n \pi}{L} \right)^2$
  \item eigenfunctions: Dirichlet $\to$ $\sin(\frac{n \pi}{L}x)$, Neumann $\to$ $\cos(\frac{n \pi}{L}x)$
  \item temporal part: heat $\to$ $e^{-\lambda_n t}$, wave $\to$ $A_n\cos(\sqrt{\lambda_n} t) + B_n\sin(\sqrt{\lambda_n} t)$
\end{itemize}

\warning{Dirichlet starts at $n=1$, Neumann at $n=0$ (constant mode for heat/wave)}

\textbf{Laplace uniqueness}: Given a bounded domain $D\subseteq \mathbb{R}^2$, the Dirichlet problem has at most one solution $u\in C^2(D)\cap C(\overline{D})$.
\[
\begin{cases}
  \Delta u = f & \text{in D}\\
  u = g' & \text{in}~\partial D
\end{cases}
\]

\subsection{non-homogenous PDEs}

\section{Elliptic equations}

When the domain $D$ has radial symmetry, polar coordinates are useful.

\[
  \begin{cases}
    x = r\cos\theta,\\
    y = r\sin\theta,
  \end{cases}
  \qquad
  A_D = \int_{0}^{\alpha}\int_{0}^{r} w(r,\theta)\, r\,dr\,d\theta.
\]

Here
\[
  r^2 = x^2 + y^2,
  \qquad
  \sin^2\theta + \cos^2\theta = 1.
\]

$$
\Delta u = W_{rr} + \frac 1r W_r + \frac{1}{r^2} W_{\theta\theta} = 0
$$

\textbf{compatibility condition (6.3.5)}: Necessary condition for $\exists$ of solution to the Neumann problem ($\Delta u = \rho$ in $D$, $\partial_\nu u = g$ on $\partial D$) is
$$
\int_{\partial D} g(x(s), y(s)) ds = \int_D \rho(x, y)\ dx\ dy
$$
with parametrization $x(s), y(s)$ of $\partial D$. Physically, the total flux through the boundary must equal the internal source.

\subsection{Harmonic Functions}

$u\in C^2(\mathbb{R}^2)$ harmonic $\iff \Delta u=0$.

A harmonic polynomial of degree $n$:
\[
  P(x,y)=\sum_{i+j\le n} a_{ij}x^iy^j,
  \qquad \Delta P=0.
\]

\[
  \begin{cases}
    n=0: & 1,\\
    n=1: & x,\ y \quad (\Rightarrow ax+by),\\
    n=2: & xy,\ x^2-y^2,\\
    n=3: & x^3-3xy^2,\ y^3-3x^2y.
  \end{cases}
\]

\section{Maximum principles}
harmonic function $u$, $\Delta u = 0$
\begin{enumerate}
  \item \textbf{Weak maximum principle:} $D$ bounded domain, $u\in C^2(D) \cap C(\overline{D})$, $u$ harmonic
    $\implies \max_{\overline{D}} u = \max_{\partial D}u$ \\
    $\implies$ the maximum of $u$ in $\overline{D}$ is achieved on the boundary $\partial D$.
  \item \textbf{Strong maximum principle:} $\Delta u = 0$ in $D$ connected \\
    if $u$ attains the maximum inside $D \implies u$ constant.
  \item \textbf{Mean-value principle:} $\Delta u = 0$, $D$ connected
\end{enumerate}
$$
u(x_0,y_0) = \frac{1}{2\pi R}\int_{\partial B_R(x_0,y_0)} u(x(s),y(s))ds = \frac{1}{2\pi}\int_0^{2\pi} u(x_0+R\cos\vartheta,\ y_0+R\sin\vartheta) d\vartheta
$$

\section{Laplace's equation in rectangular and circular domains}
\subsection{Neumann problem with rectangular domain}

\noindent
\begin{minipage}{0.49\columnwidth}
  \textbf{Compatibility condition} (necessary): Solution $\exists \iff$ net flux over boundary of domain $\Omega$ is $=0$:
  $$
  \int_{\partial\Omega} \frac{\partial u}{\partial \nu}~d\sigma = 0
  $$
  for rectangle $[a,b]\times[c,d]$:
  $$
  \int_c^d g - \int_c^d f + \int_a^b k - \int_a^b h \stackrel{!}{=} 0
  $$
\end{minipage}%
\hfill
\begin{minipage}{0.49\columnwidth}
  \centering
  \includegraphics[width=\columnwidth]{img/neumann_rectangular_problem.png}
\end{minipage}
For the Neumann boundary condition, solutions are unique up to additive constants.

\[
  \begin{cases}
    \Delta u = 0
    \quad \text{in } (a,b)\times(c,d),\\
    u_x(a,y) = f(y),\\
    u_x(b,y) = g(y),\\
    u_y(x,c) = h(x),\\
    u_y(x,d) = k(x).
  \end{cases}
\]

\textbf{Compatibility condition}
\[
  \int_c^d (g-f)\,dy + \int_a^b (k-h)\,dx = 0.
\]

\medskip

To decouple boundary conditions, add a harmonic polynomial:
\[
  v = u + \alpha(x^2 - y^2),
  \qquad \Delta(x^2-y^2)=0.
\]

Split $v=v_1+v_2$ with
\[
  \begin{cases}
    \Delta v_1 = 0,\\
    (v_1)_x(a,y) = f(y)+2\alpha a,\\
    (v_1)_x(b,y) = g(y)+2\alpha b,\\
    (v_1)_y(x,c) = 0,\\
    (v_1)_y(x,d) = 0,
  \end{cases}
  \qquad
  \begin{cases}
    \Delta v_2 = 0,\\
    (v_2)_x(a,y) = 0,\\
    (v_2)_x(b,y) = 0,\\
    (v_2)_y(x,c) = h(x)-2\alpha c,\\
    (v_2)_y(x,d) = k(x)-2\alpha d.
  \end{cases}
\]

Assuming
\[
  \int_a^b h(x)\,dx = \int_a^b k(x)\,dx = 0,
\]
the compatibility condition for $v_1$ yields
\[
  \alpha
  =
  \frac{1}{2(b-a)(d-c)}
  \int_c^d (f-g)\,dy.
\]

\bigskip

\textbf{General solutions}

{
  \scriptsize
  \[
    \textstyle
    v_1(x,y)
    =
    A_0 x + B_0
    +
    \sum_{n=1}^{\infty}
    \cos\!\left(\frac{n\pi}{d-c}(y-c)\right)
    \Bigl[
      A_n \cosh\!\left(\frac{n\pi}{d-c}(x-a)\right)
      +
      B_n \cosh\!\left(\frac{n\pi}{d-c}(x-b)\right)
    \Bigr]
  \]

  \[
    \textstyle
    v_2(x,y)
    =
    C_0 y + D_0
    +
    \sum_{n=1}^{\infty}
    \cos\!\left(\frac{n\pi}{b-a}(x-a)\right)
    \Bigl[
      C_n \cosh\!\left(\frac{n\pi}{b-a}(y-c)\right)
      +
      D_n \cosh\!\left(\frac{n\pi}{b-a}(y-d)\right)
    \Bigr]
  \]
}

\[
  u = v + \alpha(x^2-y^2).
\]

\subsection{Dirichlet problem with rectangular domain}
\begin{minipage}{0.49\columnwidth}
  \[
    \begin{cases}
      \Delta u = 0
      & (x,y) \in (a,b) \times (c,d),\\
      u(a,y) = f(y)
      & (x,y) \in \{a\} \times [c,d],\\
      u(b,y) = g(y)
      & (x,y) \in \{b\} \times [c,d],\\
      u(x,c) = h(x)
      & (x,y) \in [a,b] \times \{c\},\\
      u(x,d) = k(x)
      & (x,y) \in [a,b] \times \{d\}.
    \end{cases}
  \]
\end{minipage}%
\hfill
\begin{minipage}{0.50\columnwidth}
  \centering
  \raisebox{-0.5\height}{%
    \includegraphics[width=\columnwidth]{img/splitting_of_laplacian_rectangular_domain.png}
  }
\end{minipage}

To solve it, we split the problem into two subproblems \(u = u_1 + u_2\):

\[
  \begin{cases}
    \Delta u_1 = 0,\\
    u_1(a,y) = f(y),\\
    u_1(b,y) = g(y),\\
    u_1(x,c) = 0,\\
    u_1(x,d) = 0,
  \end{cases}
  \qquad
  \begin{cases}
    \Delta u_2 = 0,\\
    u_2(a,y) = 0,\\
    u_2(b,y) = 0,\\
    u_2(x,c) = h(x),\\
    u_2(x,d) = k(x).
  \end{cases}
\]

Separation of variables gives two ODEs for \(u_1\) (for \(u_2\) switch \(x\) and \(y\)):

\[
  X''(x) - \lambda X(x) = 0,
  \qquad
  \begin{cases}
    Y''(y) + \lambda Y(y) = 0,\\
    Y(c) = Y(d) = 0.
  \end{cases}
\]

---

{
  \scriptsize

  1. The general solution for \(u_1\) is:

  \[
    u_1(x,y)
    = \sum_{n=1}^{\infty}
    \sin\!\left( \frac{n\pi}{d-c} (y-c) \right)
    \left[
      A_n \sinh\!\left( \frac{n\pi}{d-c} (x-a) \right)
      +
      B_n \sinh\!\left( \frac{n\pi}{d-c} (x-b) \right)
    \right].
  \]

  2. The general solution for \(u_2\) is:

  \[
    u_2(x,y)
    = \sum_{n=1}^{\infty}
    \sin\!\left( \frac{n\pi}{b-a} (x-a) \right)
    \left[
      C_n \sinh\!\left( \frac{n\pi}{b-a} (y-c) \right)
      +
      D_n \sinh\!\left( \frac{n\pi}{b-a} (y-d) \right)
    \right].
  \]
}

---

3. Use the boundary conditions to determine all coefficients and:

\[
  u(x,y) = u_1(x,y) + u_2(x,y).
\]

\subsection{Disks and Annuli (Polar Coordinates)}
Let
\[
  D = \{\, (r,\theta) \mid r \in [a,b],\ \theta \in [0,2\pi) \,\}.
\]
Assume a separated solution $w(r,\theta)=R(r)\Theta(\theta)$. Then
\[
  \begin{cases}
    r^2 R''(r) + r R'(r) - \lambda R(r) = 0,\\
    \Theta''(\theta) + \lambda \Theta(\theta) = 0,\\
    \Theta(0)=\Theta(2\pi),\quad \Theta'(0)=\Theta'(2\pi).
  \end{cases}
\]

Eigenfunctions:
\[
  \Theta_n(\theta)=A_n\cos(n\theta)+B_n\sin(n\theta),
  \qquad \lambda_n=n^2,\quad n\in\mathbb{N}_0.
\]

Radial solutions:
\[
  R_n(r)=
  \begin{cases}
    C_0 + D_0 \log r, & n=0,\\[4pt]
    C_n r^n + D_n r^{-n}, & n\ge 1.
  \end{cases}
\]

\medskip

\textbf{General solution}
\[
  \begin{aligned}
    w(r,\theta)
    &=
    A_0 + B_0 \log r
    + \sum_{n=1}^{\infty} r^n
    \bigl[A_n\cos(n\theta)+B_n\sin(n\theta)\bigr] \\
    &\quad
    + \sum_{n=1}^{\infty} r^{-n}
    \bigl[C_n\cos(n\theta)+D_n\sin(n\theta)\bigr].
  \end{aligned}
\]

\medskip

\textbf{Remark.}
The terms $\log r$ and $r^{-n}$ are singular at $r=0$.
They must be omitted if $0\in D$ (disk).

\bigskip

\subsection{Disk Sectors and Annulus Sectors}

Let
\[
  D = \{\, (r,\theta) \mid r \in [a,b],\ \theta \in [0,\gamma] \,\},
  \qquad \gamma\in(0,2\pi).
\]
Assume homogeneous boundary conditions
\[
  w(r,0)=w(r,\gamma)=0.
\]

Angular problem:
\[
  \begin{cases}
    \Theta''(\theta)+\lambda\Theta(\theta)=0,\\
    \Theta(0)=\Theta(\gamma)=0,
  \end{cases}
  \quad\Rightarrow\quad
  \Theta_n(\theta)=\sin\!\left(\frac{n\pi}{\gamma}\theta\right),
  \quad
  \lambda_n=\left(\frac{n\pi}{\gamma}\right)^2.
\]

Radial solutions:
\[
  R_n(r)=C_n r^{n\pi/\gamma}+D_n r^{-n\pi/\gamma}.
\]

\medskip

\textbf{General solution}
\[
  w(r,\theta)
  =
  \sum_{n=1}^{\infty}
  \left[
    A_n r^{n\pi/\gamma}
    +
    B_n r^{-n\pi/\gamma}
  \right]
  \sin\!\left(\frac{n\pi}{\gamma}\theta\right).
\]

\medskip

\textbf{Remark.}
The term $r^{-n\pi/\gamma}$ is singular at $r=0$.
It must be omitted if $0\in D$ (disk sector).
