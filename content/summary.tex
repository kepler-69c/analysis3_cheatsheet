\section{Preliminaries}
\textbf{ODE}: Ordinary differential equation, involving functions of one independent variable and one or more of their derivatives.

\textbf{PDE}: Partial differential equation, equation involving an unknown function of more than one variable and certain of its partial derivatives.

\textbf{Schwartz}: if $u$ has continuous second partial derivatives, then the order can be changed:
$$u_{xy} = u_{yx}$$

\textbf{Gradient}: of $u = u(x, y, z)$ is defined as
$$\nabla u := (u_x, u_y, u_z)$$
and the \textbf{Laplacian} of $u$ is
$$\Delta u := u_{xx} + u_{yy} + u_{zz}$$

\textbf{Problem}: A PDE supplemented with initial or boundary conditions. A problem is \textbf{well-posed} if it satisfies the following criteria:
\begin{enumerate}
  \item The problem has a solution (existence).
  \item The solution is unique (uniqueness).
  \item A small change in the equation and/or in the side conditions gives rise to a small change in the solution (stability).
\end{enumerate}
If one or more of these conditions do not hold, then the problem is said to be \textbf{ill-posed}.

\subsection{Classification properties of PDEs}
\textbf{order} of a PDE is the order of the highest order partial derivative of the unknown appearing within it. $\implies u_{xyz} = xy^2$ has order 2

\textbf{a linear} PDE is of the form
$$
a^{(0)}u + \sum_{i_1=1}^n a^{(1)}_{i_1} u_{x_{i_1}}
+ \sum_{i_1,i_2=1}^n a^{(2)}_{i_1,i_2} u_{x_{i_1}x_{i_2}} + \dots
=:\mathcal{L}[u] = f(\mathbf{x})
$$

$\implies uu_x=2$ is not linear, but $u_{tt} = u_{xxxxx}$ is

\textbf{homogenous} PDEs have $f(x)=0$

A PDE is \textbf{quasilinear} if it is linear in its highest order derivative term.

$\implies uu_x + u^2u_y + u = e^4$ is quasilinear (of first order)

\subsection{Strong and weak solutions}
\begin{itemize}
    \item \textbf{Strong solution (or classical solution):} All derivatives that appear in the PDE exist and are continuous on the whole domain of the PDE.
    \item \textbf{Weak solution:} There are points in the domain where the derivatives do not exist (or are not continuous). A weak solution cannot directly be plugged into the equation.
\end{itemize}

$\to$ Weak solutjions are NOT unique.

\section{Method of Characteristics (MoC)}
Reduces first order / quasilinear PDEs by reducing them to a system of ODEs using geometrical interpretation.

The first two equations ($x_0(s),y_0(s)$) of the initial curve lie on the xy-plane. We use $s,t$ as a sort of "coordinate transformation"
\begin{recipe}[solve PDE using method of characteristics]
Our PDE has the form $a(x,y,u)u_x + b(x,y,u)u_y = c(x,y,u)$
\begin{enumerate}
 \item define $a$, $b$, $c$
 \item define \textbf{initial curve} based on the side condition(s)
 $$\varGamma(s) = (x_0(s), y_0(s), \tilde{u}_0(s)) = (\dots)$$
 \item solve the ODEs, using the initial values from (2):
 $$\frac{dx}{dt}=a \quad \frac{dy}{dt}=b \quad \frac{d\tilde{u}}{dt}=c$$
 \item The ODEs form a system of equations. Rearrange them according to t and s, then substitute the result into u(t) to obtain u(x, y).
\end{enumerate}
\end{recipe}


\textbf{Example} (Method of characteristics)
$$xu_x+u_y = 1 \quad u(x,0)=f(x)$$

$\to$ define $a,b,c,\varGamma$

$$
\begin{pmatrix}
a\\b\\c
\end{pmatrix} = \begin{pmatrix}
x\\1\\1
\end{pmatrix} \quad \varGamma(s) = \begin{pmatrix}
x_{0}(s)\\y_{0}(s) \\\tilde{u}_{0}(s)
\end{pmatrix} = \begin{pmatrix}
s\\0\\f(s)
\end{pmatrix}
$$

$\to$ solve ODEs

\begin{align*}
 \frac{dx}{dt} = x & &  \implies & x(t) = C_{1}e^t \\
 \text{initial condition:}  & &  & x(0) = C_{1} \stackrel{!}{=} s \\
 &  & \implies & x(s) = se^t \\
\frac{dy}{dt} = 1 &  & \implies  & y(t) = t+C_{2} \\
\text{initial condition:}  &  &  & y(0) = C_{2} \stackrel{!}{=} 0 \\
 &  & \implies & y(t) = t \\
\frac{d\tilde{u}}{dt} = 1 &  & \implies & \tilde{u}(t) = t+C_{3} \\
\text{initial condition:} &  &  & \tilde{u}(0) = C_{3} \stackrel{!}{=} f(s) \\
 &  & \implies & \tilde{u}(t) = t + f(s)
\end{align*}

$\to$ rearrange to x,y

$$
y=t;\quad\quad x=se^t = se^y \iff s=xe^{-y};\quad\quad \tilde{u}(t) = t+f(s) = y + f(xe^{-y})
$$

Thus, the result is $u(x,y) = y + f(xe^{-y})$.

\subsection{Existence and uniqueness}

\section{Conservation laws and shock waves}

\textbf{conservation law} ($f:\mathbb{R}\to\mathbb{R}$ flux function, $c(u)=f'(u)$)
$$u_y + f(u)_x = 0 \quad \text{or} \quad u_y + c(u)u_x = 0$$

\textbf{shock wave}: forms when characteristics intersect. Speed given by \textbf{Rankine-Hugoniot}:
$$\sigma'(y) = \frac{f(u^+) - f(u^-)}{u^+ - u^-}$$

\textbf{critical time} (note that $c(u)=f'(u)$):
$$
y_c = \inf\left\{ \frac{-1}{c'(u_0(s))u_0'(s)}: \quad s\in \mathbb{R},\ c'(u_{0}(s))u_{0}'(s)<0 \right\}
$$

A function is a weak solution $\iff$ \textbf{integral formulation} of conservation laws holds $\forall a<b,\ y_1<y_2$:
$$
\int_{a}^{b} u(x, y_{2}) \,dx - \int_{a}^{b} u(x, y_{1}) \,dx = - \int_{y_{1}}^{y_{2}} \left[ f(u(b, y)) - f(u(a, y)) \right] \,dy
$$

\textbf{entropy condition}: A weak solution satifies the entropy condition of characteristics only enter shocks but do not emanate from them This means:
$$c(u^+) < \sigma' < c(u^-)$$

(this condition ensures uniqueness of weak solution)

\subsection{2nd order linear PDE classification}
\hl{TODO}: where to put this subsection?

All second-order linear PDEs can be written in the general form:
$$
L[u] = a u_{xx} + 2b u_{xy} + c u_{yy} + d u_x + e u_y + f u = g.
$$
Where $a, b, c, d, e, f, g$ are all functions of $(x, y)$ but not $u$.

The \textbf{discriminant} $\delta(x_0, y_0)$ is defined as
$$
\delta(x, y) := b^2 - a c.
$$

The PDE is classified based on the sign of the discriminant $\delta$:
\begin{itemize}
    \item If $\delta > 0$: PDE is \textbf{hyperbolic} ($\sim$ Wave equation)
    \item If $\delta = 0$: PDE is \textbf{parabolic} ($\sim$ Heat equation)
    \item If $\delta < 0$: PDE is \textbf{elliptic} ($\sim$ Laplace equation)
\end{itemize}

\section{One dimensional wave equation}
Hyperbolic 2nd order differential equation of the form:
$$
\begin{cases}
 u_{tt} - c^2u_{xx} &= F(x,t) \\
 u(x,0) &= f(x) \\
 u_{t}(x,0) &= g(x)
\end{cases}
$$

\subsection{homogenous wave equation ($F(x,t)=0$)}
Solution consists of a backwards-moving wave $F$ and forward-moving wave $G$:
$$
u(x,t) = F(x+ct) + G(x-ct)
$$
The solution can be calculated using d'Alembert's (short) formula:
$$
u(x,t) = \frac{f(x+ct) + f(x-ct)}{2} + \frac{1}{2c} \int_{x-ct}^{x+ct}g(y)\ dy
$$

\subsection{non-homogenous wave equation}
D'Alembert's formula for non-homogenous wave equations:
$$
u(x,t) = \frac{f(x+ct) + f(x-ct)}{2} + \frac{1}{2c} \int_{x-ct}^{x+ct}g(y)\ dy + \frac{1}{2c}\int_0^t \int_{x-c(t-\tau)}^{x+c(t-\tau)} F(\xi, \tau) \ d\xi d\tau
$$

\begin{recipe}[trick: converting to homogenous wave equation]
For a non-homogenous wave equation $u$, define $w:= u-v$, $w$ homogenous:
$$
\begin{cases}
w_{tt}-c^2 u_{xx}  & =0 \\
w(x,0) = u(x,0)-v(x,0)  & = f(x) - v(x,0) \\
w_{t}(x,0) = u_{t}(x,0) - v_{t}(x,0)  & = g(x) - v_{t}(x,0)
\end{cases}
$$
After finding $v$, we can now solve $w$ using d'Alembert's simple formula.

\hl{TODO}: How to find $v$?
\end{recipe}
\section{Seperation of variables}
Can be applied to "nearly all" linear 2nd-order PDEs

PDE $\leadsto$ 2 ODE systems $\leadsto$ boundary conditions to find eigenvalues $\lambda$, eigenfunctions $\sin/\hdots \leadsto$ initial condition

\begin{recipe}[Apply seperation of variables]
Assuming a product solution $u(x,t) = X(x)T(t)$
\begin{enumerate}
 \item substite into PDE, separating variables
 \item set both sides = const. $-\lambda \leadsto$ two ODEs:

  $$\text{\hl{TODO}}$$

 \item solve each ODE individually for $X$ and $T$
 \item apply boundary conditions \hl{TODO}: how?
 \item apply initial condition to get coefficients of resulting (fourier series) solution \hl{TODO}: maybe add example
\end{enumerate}
\end{recipe}

\hl{TODO}: \textbf{are the following formulas relevant?}

General solution for heat equation with Dirichlet b.c.

\hl{TODO}

General solution for heat equation with Neumann b.c.

\hl{TODO}

General solution for wave equation with Neumann b.c.
$$
u(x,t) = \sum_{n=0}^{+\infty} X_n(x)T_n(t) = \frac{A_0+B_0 t}{2} + \sum_{n=1}^{+\infty} cos(\frac{n\pi x}{L}\left[ A_n cos(\frac{n\pi c}{L}t) + sin(\frac{n\pi c}{L}t)B_n \right])
$$

where $\frac{n\pi}{L}=\lambda$, $\cos$ comes from Neumann / $X(x)$, $[\dots]$ from $T(t)$

\subsection{Boundary Conditions}
\begin{itemize}
    \item Dirichlet: $u(0,t) = u(L,t) = 0$.
    \item Neumann: $u_x(0,t) = u_x(L,t) = 0$. ("no flux")
    \item Mixed type, or Robin:
    $$\alpha_0 u(0,t) + \beta_0 u_x(0,t) = \gamma_0, \quad \alpha_L u(L,t) + \beta_L u_x(L,t) = \gamma_L$$
\end{itemize}

\warning{Dirichlet $\implies$ only $\sin$ in general solution; Neumann $\implies$ only $\cos$}

\section{Elliptic equations}

\section{Maximum principles}

\section{Laplace's equation in rectangular and circular domains}
