\section{Preliminaries}
\textbf{ODE}: Ordinary differential equation, involving functions of one independent variable and one or more of their derivatives.

\textbf{PDE}: Partial differential equation, equation involving an unknown function of more than one variable and certain of its partial derivatives.

\textbf{Schwartz}: if $u$ has continuous second partial derivatives, then the order can be changed:
$$u_{xy} = u_{yx}$$

\textbf{Gradient}: of $u = u(x, y, z)$ is defined as
$$\nabla u := (u_x, u_y, u_z)$$
and the \textbf{Laplacian} of $u$ is
$$\Delta u := u_{xx} + u_{yy} + u_{zz}$$

\textbf{Problem}: A PDE supplemented with initial or boundary conditions. A problem is \textbf{well-posed} if it satisfies the following criteria:
\begin{enumerate}
  \item The problem has a solution (existence).
  \item The solution is unique (uniqueness).
  \item A small change in the equation and/or in the side conditions gives rise to a small change in the solution (stability).
\end{enumerate}
If one or more of these conditions do not hold, then the problem is said to be \textbf{ill-posed}.

\subsection{Classification properties of PDEs}
\textbf{order} of a PDE is the order of the highest order partial derivative of the unknown appearing within it.

$\implies u_{xyz} = xy^2$ has order 2 

\textbf{a linear} PDE is of the form
$$
a^{(0)}u + \sum_{i_1=1}^n a^{(1)}_{i_1} u_{x_{i_1}}
+ \sum_{i_1,i_2=1}^n a^{(2)}_{i_1,i_2} u_{x_{i_1}x_{i_2}} + \dots
=:\mathcal{L}[u] = f(\mathbf{x})
$$

$\implies uu_x=2$ is not linear, but $u_{tt} = u_{xxxxx}$ is

\textbf{homogenous} PDEs have $f(x)=0$

A PDE is \textbf{quasilinear} if it is linear in its highest order derivative term.

$\implies uu_x + u^2u_y + u = e^4$ is quasilinear (of first order)

\section{Method of characteristics}
Reduces first order / quasilinear PDEs by reducing them to a system of ODEs using geometrical interpretation.

\begin{recipe}[solve PDE using method of characteristics]
Our PDE has the form $a(x,y,u)u_x + b(x,y,u)u_y = c(x,y,u)$
\begin{enumerate}
 \item define $a$, $b$, $c$
 \item define $\varGamma(s) = (x_0(s), y_0(s), u_0(s)) = (\dots)$ based on the side condition
 \item solve the ODEs, using the initial values from (2):
 $$\frac{dx}{dt}=a \quad \frac{dy}{dt}=b \quad \frac{du}{dt}=c$$
 \item The ODEs form a system of equations. Rearrange them according to t and s, then substitute the result into u(t) to obtain u(x, y).
\end{enumerate}
\end{recipe}

\textbf{Example} (Method of characteristics)
$$xu_x+u_y = 1 \quad u(x,0)=f(x)$$

$\to$ define $a,b,c,\varGamma$

$$
\begin{pmatrix}
a\\b\\c
\end{pmatrix} = \begin{pmatrix}
x\\1\\1
\end{pmatrix} \quad \varGamma(s) = \begin{pmatrix}
x_{0}(s)\\y_{0}(s) \\u_{0}(s)
\end{pmatrix} = \begin{pmatrix}
s\\0\\f(s)
\end{pmatrix}
$$

$\to$ solve ODSs

\begin{align*}
 \frac{dx}{dt} = x & &  \implies & x(t) = C_{1}e^t \\
 \text{initial condition:}  & &  & x(0) = C_{1} \stackrel{!}{=} s \\
 &  & \implies & x(s) = se^t \\
\frac{dy}{dt} = 1 &  & \implies  & y(t) = t+C_{2} \\
\text{initial condition:}  &  &  & y(0) = C_{2} \stackrel{!}{=} 0 \\
 &  & \implies & y(t) = t \\
\frac{du}{dt} = 1 &  & \implies & u(t) = t+C_{3} \\
\text{initial condition:} &  &  & u(0) = C_{3} \stackrel{!}{=} f(s) \\
 &  & \implies & u(t) = t + f(s)
\end{align*}

$\to$ rearrange to x,y

$$
y=t;\quad\quad x=se^t = se^y \iff s=xe^{-y};\quad\quad u(t) = t+f(s) = y + f(xe^{-y})
$$

Thus, the result is $u(x,y) = y + f(xe^{-y})$.
