\section{Preliminaries}
\textbf{ODE}: Ordinary differential equation, involving functions of one independent variable and one or more of their derivatives.

\textbf{PDE}: Partial differential equation, equation involving an unknown function of more than one variable and certain of its partial derivatives.

\textbf{Schwartz}: if $u$ has continuous second partial derivatives, then the order can be changed:
$$u_{xy} = u_{yx}$$

\textbf{Gradient}: of $u = u(x, y, z)$ is defined as
$$\nabla u := (u_x, u_y, u_z)$$
and the \textbf{Laplacian} of $u$ is
$$\Delta u := u_{xx} + u_{yy} + u_{zz}$$

\textbf{Problem}: A PDE supplemented with initial or boundary conditions. A problem is \textbf{well-posed} if it satisfies the following criteria:
\begin{enumerate}
  \item The problem has a solution (existence).
  \item The solution is unique (uniqueness).
  \item A small change in the equation and/or in the side conditions gives rise to a small change in the solution (stability).
\end{enumerate}
If one or more of these conditions do not hold, then the problem is said to be \textbf{ill-posed}.

\subsection{Classification properties of PDEs}
\textbf{order} of a PDE is the order of the highest order partial derivative of the unknown appearing within it. $\implies u_{xyz} = xy^2$ has order 2

\textbf{a linear} PDE is of the form
$$
a^{(0)}u + \sum_{i_1=1}^n a^{(1)}_{i_1} u_{x_{i_1}}
+ \sum_{i_1,i_2=1}^n a^{(2)}_{i_1,i_2} u_{x_{i_1}x_{i_2}} + \dots
=:\mathcal{L}[u] = f(\mathbf{x})
$$

$\implies uu_x=2$ is not linear, but $u_{tt} = u_{xxxxx}$ is

\textbf{homogenous} PDEs have $f(x)=0$

A PDE is \textbf{quasilinear} if it is linear in its highest order derivative term.

$\implies uu_x + u^2u_y + u = e^4$ is quasilinear (of first order)

\subsection{Strong and weak solutions}
\begin{itemize}
    \item \textbf{Strong solution (or classical solution):} All derivatives that appear in the PDE exist and are continuous on the whole domain of the PDE.
    \item \textbf{Weak solution:} There are points in the domain where the derivatives do not exist (or are not continuous). A weak solution cannot directly be plugged into the equation.
\end{itemize}

$\to$ Weak solutjions are NOT unique.

\section{Method of Characteristics (MoC)}
Reduces first order / quasilinear PDEs by reducing them to a system of ODEs using geometrical interpretation.

The first two equations ($x_0(s),y_0(s)$) of the initial curve lie on the xy-plane. We use $s,t$ as a sort of "coordinate transformation"
\begin{recipe}[solve PDE using method of characteristics]
Our PDE has the form $a(x,y,u)u_x + b(x,y,u)u_y = c(x,y,u)$
\begin{enumerate}
 \item define $a$, $b$, $c$
 \item define \textbf{initial curve} based on the side condition(s)
 $$\varGamma(s) = (x_0(s), y_0(s), \tilde{u}_0(s)) = (\dots)$$
 \item solve the ODEs, using the initial values from (2):
 $$\frac{dx}{dt}=a \quad \frac{dy}{dt}=b \quad \frac{d\tilde{u}}{dt}=c$$
 \item The ODEs form a system of equations. Rearrange them according to t and s, then substitute the result into u(t) to obtain u(x, y).
\end{enumerate}
\end{recipe}


\textbf{Example} (Method of characteristics)
$$xu_x+u_y = 1 \quad u(x,0)=f(x)$$

$\to$ define $a,b,c,\varGamma$

$$
\begin{pmatrix}
a\\b\\c
\end{pmatrix} = \begin{pmatrix}
x\\1\\1
\end{pmatrix} \quad \varGamma(s) = \begin{pmatrix}
x_{0}(s)\\y_{0}(s) \\\tilde{u}_{0}(s)
\end{pmatrix} = \begin{pmatrix}
s\\0\\f(s)
\end{pmatrix}
$$

$\to$ solve ODEs

\begin{align*}
 \frac{dx}{dt} = x & &  \implies & x(t) = C_{1}e^t \\
 \text{initial condition:}  & &  & x(0) = C_{1} \stackrel{!}{=} s \\
 &  & \implies & x(s) = se^t \\
\frac{dy}{dt} = 1 &  & \implies  & y(t) = t+C_{2} \\
\text{initial condition:}  &  &  & y(0) = C_{2} \stackrel{!}{=} 0 \\
 &  & \implies & y(t) = t \\
\frac{d\tilde{u}}{dt} = 1 &  & \implies & \tilde{u}(t) = t+C_{3} \\
\text{initial condition:} &  &  & \tilde{u}(0) = C_{3} \stackrel{!}{=} f(s) \\
 &  & \implies & \tilde{u}(t) = t + f(s)
\end{align*}

$\to$ rearrange to x,y

$$
y=t;\quad\quad x=se^t = se^y \iff s=xe^{-y};\quad\quad \tilde{u}(t) = t+f(s) = y + f(xe^{-y})
$$

Thus, the result is $u(x,y) = y + f(xe^{-y})$.

\subsection{Existence and uniqueness}

\section{Conservation laws and shock waves}

\textbf{conservation law} ($f:\mathbb{R}\to\mathbb{R}$ flux function, $c(u)=f'(u)$)
$$u_y + f(u)_x = 0 \quad \text{or} \quad u_y + c(u)u_x = 0$$

\textbf{shock wave}: forms when characteristics intersect. Speed given by \textbf{Rankine-Hugoniot}:
$$\sigma'(y) = \frac{f(u^+) - f(u^-)}{u^+ - u^-}$$

\textbf{critical time} (note that $c(u)=f'(u)$):
$$
y_c = \inf\left\{ \frac{-1}{c'(u_0(s))u_0'(s)}: \quad s\in \mathbb{R},\ c'(u_{0}(s))u_{0}'(s)<0 \right\}
$$

A function is a weak solution $\iff$ \textbf{integral formulation} of conservation laws holds $\forall a<b,\ y_1<y_2$:
$$
\int_{a}^{b} u(x, y_{2}) \,dx - \int_{a}^{b} u(x, y_{1}) \,dx = - \int_{y_{1}}^{y_{2}} \left[ f(u(b, y)) - f(u(a, y)) \right] \,dy
$$

\textbf{entropy condition}: A weak solution satifies the entropy condition of characteristics only enter shocks but do not emanate from them This means:
$$c(u^+) < \sigma' < c(u^-)$$

(this condition ensures uniqueness of weak solution)

\subsection{2nd order linear PDE classification}
\hl{TODO}: where to put this subsection?

All second-order linear PDEs can be written in the general form:
$$
L[u] = a u_{xx} + 2b u_{xy} + c u_{yy} + d u_x + e u_y + f u = g.
$$
Where $a, b, c, d, e, f, g$ are all functions of $(x, y)$ but not $u$.

The \textbf{discriminant} $\delta(x_0, y_0)$ is defined as
$$
\delta(x, y) := b^2 - a c.
$$

The PDE is classified based on the sign of the discriminant $\delta$:
\begin{itemize}
    \item If $\delta > 0$: PDE is \textbf{hyperbolic} ($\sim$ Wave equation)
    \item If $\delta = 0$: PDE is \textbf{parabolic} ($\sim$ Heat equation)
    \item If $\delta < 0$: PDE is \textbf{elliptic} ($\sim$ Laplace equation)
\end{itemize}

\section{One dimensional wave equation}
Hyperbolic 2nd order differential equation of the form:
$$
\begin{cases}
 u_{tt} - c^2u_{xx} &= F(x,t) \\
 u(x,0) &= f(x) \\
 u_{t}(x,0) &= g(x)
\end{cases}
$$

\subsection{homogenous wave equation ($F(x,t)=0$)}
Solution consists of a backwards-moving wave $F$ and forward-moving wave $G$:
$$
u(x,t) = F(x+ct) + G(x-ct)
$$
The solution can be calculated using d'Alembert's (short) formula:
$$
u(x,t) = \frac{f(x+ct) + f(x-ct)}{2} + \frac{1}{2c} \int_{x-ct}^{x+ct}g(y)\ dy
$$

\subsection{non-homogenous wave equation}
D'Alembert's formula for non-homogenous wave equations:
$$
u(x,t) = \frac{f(x+ct) + f(x-ct)}{2} + \frac{1}{2c} \int_{x-ct}^{x+ct}g(y)\ dy + \frac{1}{2c}\int_0^t \int_{x-c(t-\tau)}^{x+c(t-\tau)} F(\xi, \tau) \ d\xi d\tau
$$

\begin{recipe}[trick: converting to homogenous wave equation]
For a non-homogenous wave equation $u$, define $w:= u-v$, $w$ homogenous:
$$
\begin{cases}
w_{tt}-c^2 u_{xx}  & =0 \\
w(x,0) = u(x,0)-v(x,0)  & = f(x) - v(x,0) \\
w_{t}(x,0) = u_{t}(x,0) - v_{t}(x,0)  & = g(x) - v_{t}(x,0)
\end{cases}
$$
After finding $v$, we can now solve $w$ using d'Alembert's simple formula.

\hl{TODO}: How to find $v$?
\end{recipe}

\textbf{Example: 1D Wave Equation on a Ring (periodic extension)}

\[
\begin{cases}
u_{tt} - c^2 u_{xx} = 0, & x \in [0,1],\, t>0,\\[0.3em]
u(x,0) = v_0(x), \quad u_t(x,0) = v_1(x),\\[0.3em]
u(0,t) = u(1,t), \quad u_x(0,t) = u_x(1,t) \quad (\text{periodic BC})
\end{cases}
\]

\[
v_0(x) = x - x^2, \quad v_1(x) = 0, \quad c = \sqrt{\pi} - 1
\]

\vspace{0.3em}
\textit{D’Alembert solution:}
\[
u(x,t) = \tfrac{1}{2}\big[v_0(x+ct) + v_0(x-ct)\big]
      + \tfrac{1}{2c}\int_{x-ct}^{x+ct} v_1(y)\,dy
\]

Since \(v_1(x)=0\):
\[
u(x,t) = \tfrac{1}{2}\big[v_0(x+ct) + v_0(x-ct)\big]
\]

\textit{At } \(x=\tfrac{1}{2},\; t=2025\):
\[
u\!\left(\tfrac{1}{2},2025\right)
 = \tfrac{1}{2}\big[v_0(\tfrac{1}{2}+t) + v_0(\tfrac{1}{2}-t)\big]
 = v_0\!\left(\tfrac{1}{2}\right)
 = \tfrac{1}{2} - \left(\tfrac{1}{2}\right)^2
 = \tfrac{1}{4}.
\]


\subsection{Uniqueness}
The 1D-wave equation has a unique solution

\subsection{Symmetry}
If the initial data $f_{x}$, $g_{x}$ and the inhomogenity $F_{x,t}$ are even/odd/periodic with respect to x, then the solution is even/odd/periodic with respect to x as well


\section{Domain of dependence and region of influence}

\hl{TODO}: \textbf{Add short explanation (with picture from lecture notes)}

\section{Seperation of variables}
Can be applied to "nearly all" linear 2nd-order PDEs

PDE $\leadsto$ 2 ODE systems $\leadsto$ boundary conditions to find eigenvalues $\lambda$, eigenfunctions $\sin/\hdots \leadsto$ initial condition

\begin{recipe}[Apply seperation of variables]
Assuming a product solution $u(x,t) = X(x)T(t)$
\begin{enumerate}
 \item substite into PDE, separating variables
 \item set both sides = const. $-\lambda \leadsto$ two ODEs:

  $$\text{\hl{TODO}}$$

 \item solve each ODE individually for $X$ and $T$
 \item apply boundary conditions \hl{TODO}: how?
 \item apply initial condition to get coefficients of resulting (fourier series) solution \hl{TODO}: maybe add example
\end{enumerate}
\end{recipe}

\hl{TODO}: \textbf{are the following formulas relevant?}

General solution for heat equation with Dirichlet b.c.

\hl{TODO}

General solution for heat equation with Neumann b.c.

\hl{TODO}

General solution for wave equation with Neumann b.c.
$$
u(x,t) = \sum_{n=0}^{+\infty} X_n(x)T_n(t) = \frac{A_0+B_0 t}{2} + \sum_{n=1}^{+\infty} cos(\frac{n\pi x}{L}\left[ A_n cos(\frac{n\pi c}{L}t) + sin(\frac{n\pi c}{L}t)B_n \right])
$$

where $\frac{n\pi}{L}=\lambda$, $\cos$ comes from Neumann / $X(x)$, $[\dots]$ from $T(t)$

\subsection{Boundary Conditions}
\begin{itemize}
    \item Dirichlet: $u(0,t) = u(L,t) = 0$.
    \item Neumann: $u_x(0,t) = u_x(L,t) = 0$. ("no flux")
    \item Mixed type, or Robin:
    $$\alpha_0 u(0,t) + \beta_0 u_x(0,t) = \gamma_0, \quad \alpha_L u(L,t) + \beta_L u_x(L,t) = \gamma_L$$
\end{itemize}

\warning{Dirichlet $\implies$ only $\sin$ in general solution; Neumann $\implies$ only $\cos$}

\section{Elliptic equations}

\section{Maximum principles}

\section{Laplace's equation in rectangular and circular domains}



\section{Appendix}

% Fand ich noch praktisch, hab ich von einer anderen ZF kopiert und ergänzt von unserer vorherigen ZF von Anal 1/2
\subsection{Trigonometry}
\begin{center}
    $\sinh(x)$ is odd and $\cosh(x)$ is even!
\end{center}

\begin{center}
    \renewcommand{\arraystretch}{1.5}
    \begin{tabular}{l l} \toprule
        \multicolumn{2}{l}{$\cos(x\pm y)=\cos(x)\cos(y)\mp\sin(x)\sin(y)$}                                                  \\
        \multicolumn{2}{l}{$\sin(x\pm y)=\sin(x)\cos(y)\pm\cos(x)\sin(y)$}                                                  \\
        \midrule
        $\cos \left(x+\frac{1}{2} \pi\right)=-\sin (x)$  & \hspace*{-10pt} $\sin \left(x+\frac{1}{2} \pi\right)=\cos (x)$   \\
        \midrule
        \multicolumn{2}{l}{$\sin(x)\sin(y)=\frac{1}{2}(\cos(x-y)-\cos(x+y))$}                                               \\
        \multicolumn{2}{l}{$\cos(x)\cos(y)=\frac{1}{2}(\cos(x-y)+\cos(x+y))$}                                               \\
        \multicolumn{2}{l}{$\sin(x)\cos(y)=\frac{1}{2}(\sin(x-y)+\sin(x+y))$}                                               \\
            \midrule
        $\sin^2(x)=\frac{1}{2}(1-\cos(2x))$              & \hspace*{-10pt} $\cos^2(x)=\frac{1}{2}(1+\cos(2x))$              \\
        $\sin^{3}(x)=\frac{1}{4}(3 \sin (x)-\sin (3 x))$ & \hspace*{-10pt} $\cos^{3}(x)=\frac{1}{4}(3 \cos (x)+\cos (3 x))$ \\
        \bottomrule
    \end{tabular}
\end{center}

\begin{center} 
    \begin{minipage}{0.47 \linewidth}
        \begin{center}
            \begin{tabular}{r l}
                $\sin(z)$ & \hspace*{-10pt}$= \dfrac{e^{iz} - e^{-iz}}{2i}$ \\[4mm]
                $\cos(z)$ & \hspace*{-10pt}$= \dfrac{e^{iz} + e^{-iz}}{2}$  \\
            \end{tabular}
        \end{center}
    \end{minipage}
    \begin{minipage}{0.47 \linewidth}
        \begin{center}
            \begin{tabular}{r l}
                $\sinh(z)$ & \hspace*{-10pt}$= \dfrac{e^{z} - e^{-z}}{2}$ \\[4mm]
                $\cosh(z)$ & \hspace*{-10pt}$= \dfrac{e^{z} + e^{-z}}{2}$ \\
            \end{tabular}
        \end{center}
    \end{minipage}
\end{center}

\begin{center}
    \renewcommand{\arraystretch}{1.25}
    \begin{tabular}{r c c c c c} \toprule
        deg/rad & 0$^\circ$/0 & 30$^\circ$/$\frac{\pi}{6}$ & $45^\circ$/$\frac{\pi}{4}$ & 60$^\circ$/$\frac{\pi}{3}$ & 90$^\circ$/$\frac{\pi}{2}$ \\ \midrule
        sin     & $0$         & $\frac{\sqrt{1}}{2}$       & $\frac{\sqrt{2}}{2}$       & $\frac{\sqrt{3}}{2}$       & $1$                        \\
        cos     & $1$         & $\frac{\sqrt{3}}{2}$       & $\frac{\sqrt{2}}{2}$       & $\frac{\sqrt{1}}{2}$       & $0$                        \\
        tan     & $0$         & $\frac{\sqrt{3}}{3}$       & $1$                        & $\sqrt{3}$                 & -                          \\ \bottomrule
    \end{tabular}
    \begin{tabular}{r c c c c} \toprule
        deg/rad & 120$^\circ$/$\frac{2\pi}{3}$ & 135$^\circ$/$\frac{3\pi}{4}$ & $150^\circ$/$\frac{5\pi}{6}$ & 180$^\circ$/$\pi$ \\ \midrule
        sin     & $\frac{\sqrt{3}}{2}$         & $\frac{\sqrt{2}}{2}$         & $\frac{1}{2}$                & $0$               \\
        cos     & $- \frac{1}{2}$              & $- \frac{\sqrt{2}}{2}$       & $- \frac{\sqrt{3}}{2}$       & $-1$              \\
        tan     & $-\sqrt{3}$                  & $- 1$         & $-\frac{\sqrt{3}}{3}$        & $0$               \\ \bottomrule
    \end{tabular}
\end{center}


\subsection{Derivatives and Antiderivatives}

\begin{center}
\renewcommand{\arraystretch}{1.3}
\begin{tabular}{|>{$}c<{$}|>{$}c<{$}|>{$}c<{$}|}
\hline
f'(x) & f(x) & \int f(x)\,dx \\
\hline
0 & c & cx \\
nx^{n-1} & x^n & \frac{x^{n+1}}{n+1} \\
-\frac{1}{x^2} & \frac{1}{x} & \ln|x| \\
\frac{n}{x^{n+1}} & \frac{1}{x^n} & \frac{-1}{(n-1)x^{n-1}} \\
\frac{1}{2\sqrt{x}} & \sqrt{x} & \frac{2}{3}x^{3/2} \\
ae^{ax} & e^{ax} & \frac{1}{a}e^{ax} \\
\frac{1}{x} & \ln|x| & x(\ln x - 1) \\
a^x \ln a & a^x & \frac{1}{\ln a} a^x \\
\cos(x) & \sin(x) & -\cos(x) \\
-\sin(x) & \cos(x) & \sin(x) \\
\frac{1}{\cos^2(x)} & \tan(x) & -\ln|\cos(x)| \\
\cosh(x) & \sinh(x) & \cosh(x) \\
\sinh(x) & \cosh(x) & \sinh(x) \\
\frac{1}{\cosh^2(x)} & \tanh(x) & \ln(\cosh(x)) \\
\frac{1}{1+x^2} & \arctan(x) & x\arctan(x) - \frac{\ln(x^2+1)}{2}\\
\frac{1}{\sqrt{1 - x^2}} & \arcsin(x) & x \arcsin(x) + \sqrt{1-x^2}\\
-\frac{1}{\sqrt{1 - x^2}} & \arccos(x) & x \arccos(x) - \sqrt{1-x^2}\\
\frac{1}{\sqrt{1 + x^2}} & \text{arcsinh}(x) & x \, \text{arcsinh}(x) - \sqrt{x^2+1} \\
\frac{1}{\sqrt{x^2 - 1}} \quad x \geq 1 & \text{arccosh}(x) & x \, \text{arccosh}(x) - \sqrt{x^2 - 1} \\
\frac{1}{1 - x^2} \quad |x| < 1 & \text{artanh}(x) & x \, \text{artanh}(x) + \frac{1}{2} \ln(1 - x^2) \\
\frac{1}{m((x+k)^2 + m^2)} & \frac{1}{(x+k)^2 + m^2} & \frac{1}{m} \arctan\left(\frac{x + k}{m}\right) \\
\hline
\end{tabular}
\end{center}